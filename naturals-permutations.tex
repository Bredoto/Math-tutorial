\section{Перестановки}

Мы часто переставляем физические предметы местами: тасуем карты, перекладываем деньги в кошельке, сортируем данные в компьютере, ставим книги на полке и так далее. Эти действия отражены в математическом понятии перестановок, которым мы и займёмся. Чтобы не было путаницы, мы более не будем придерживаться аксиоматического задания натуральных чисел (если не оговорено обратное), и в этом и последующих параграфах мы будем использовать обозначение $[n]$ для множества чисел от 1 до $n$.

\begin{definition}
\term{$n$-элементрной перестановкой} называется биекция $[n]\to[n]$. Множество всех $n$-элементных перестановок будем обозначать как $S_n$.
\end{definition}

Перестановку удобно записывать в виде строки. Например: $\pi = 35124 \in S_5$. Эта запись означает, что на первой позиции оказался элемент, который раньше стоял на третьей позиции, вслед за ним идёт элемент, которые ранее стоял пятым и т.д.

Как и для любых функций, для перестановок можно рассматривать их композицию (в случае перестановок её часто называют \term{умножением}). Пусть, например, перестановка $\pi$ задана как выше и дана перестановка $\rho = 45231$. В этом случае мы получаем $\pi \circ \rho = 21534$. Для понимания этого примера важно помнить, что $(\pi\circ\rho)(x) = \pi(\rho(x))$, то есть для того, чтобы получить позицию элемента, надо вначале применить перестановку $\rho$, а затем только $\pi$. Такой <<обратный порядок>> применения перестановок может показаться непривычным, но это стандартная форма записи функций, как мы узнали во второй главе. Вот так например можно получить позицию единицы в искомой перестановке:

$$(\pi\circ\rho) (1) = \pi(\rho(1)) = \pi(5) = 2$$

Из того, что перестановка~--- это функция, сразу следует ассоциативность перемножения перестановок (см.~\S2.4): для любых $\rho, \pi, \delta$ мы имеем тождество
$$(\rho\circ\pi)\circ\delta = \rho\circ(\pi\circ\delta) = \rho\circ\pi\circ\delta$$

\begin{exercise}
Покажите, что умножение перестановок не коммутативно, то есть что в общем случае $\rho\circ\pi \not= \pi\circ\rho$
\end{exercise}

\begin{exercise}
Придумайте сами каких-нибудь перестановок и поэкспериментируйте с ними.
\end{exercise}

\begin{thm}
$(\rho\circ\pi)^{-1} = \pi^{-1}\circ\rho^{-1}$
\end{thm}
\begin{proof}
$$(\rho\circ\pi)\circ(\pi^{-1}\circ\rho^{-1}) = \rho\circ(\pi\circ\pi^{-1})\circ\rho^{-1} = \rho\circ\rho^{-1} = 1_{[n]}$$
\end{proof}

Если вам здесь что-то стало не очень понятно или сложно~--- перечитайте~\S2.4, все обозначения и термины являются непосредственным отражением того, что обсуждалось когда мы говорили об обыкновенных функциях.

Подсчитаем мощность множества $S_n$. На первую позицию мы можем поставить один из $n$ доступных нам элементов. На вторую позицию мы можем выбрать один из оставшихся $n-1$ элементов (итого способов расставить первые два элемента $n(n-1)$). Затем на третью позицию мы можем поставить один из $n-2$ элементов (итого имеем $n(n-1)(n-2)$ расстановок). Продолжая так до конца, на последнюю $n$-ю позицию мы ставим один оставшийся элемент. Для формулировки полученного результата полезно следующее обозначение:

\begin{definition}
Факториалом $n!$ называется величина $$n(n-1)(n-2)\ldots 2\cdot 1$$
Так же считаем, что $0! = 1$.
\end{definition}

Значение для $0!$ оправдано с той точки зрения, что на пустом множестве формально можно задать единственную функцию, которая так же будет биекцией (хоть она ничего и не отображает, формально она есть; см. аналогию с $n^0$ в~\S1).

Рассуждения, приведенные выше, теперь можно сформулировать таким образом:

\begin{thm}
$|S_n| = n!$
\end{thm}

Помимо представления перестановки в виде строки, их удобно рассматривать в виде \term{циклов}. Рассмотрим опять перестановку $\rho = 45231$. Мы видим, что при этой перестановке 1 переходит на позицию 5, 5 переходит на позицию 2, 2 на позицию 3, 3 на 4, которая переходит опять на позицию 1. Эти переходы записываются строкой чисел в круглых скобках: $\rho = (15234)$. Здесь за каждым числом $x$ число $\rho(x)$. Последнее значение переходит на позицию, обозначенную первым числом, что замыкает цикл.

Рассмотрим теперь перестановку $\pi = (35124)$. Здесь $\pi(1) = 3$ и $\pi(3) = 1$, что даёт цикл $(13)$. Сюда, однако вошли не все элементы множества. Рассмотрим оставшиеся элементы: $\pi(2) = 4$, $\pi(4) = 5$, $\pi(5) = 2$, что даёт цикл $(245)$. Итого перестановка представляется в виде произведения двух циклов: $\pi = (13)(245)$.

\begin{definition}
Циклической перестановкой $[n]$ называется перестановка, состоящая из единственного цикла длины $n$.
\end{definition}

\begin{thm}
Существует $(n-1)!$ циклических перестановок множества $[n]$.
\end{thm}
\begin{proof}
Чтобы получить циклическую перестановку $[n]$ нам достаточно записать числа от 1 до $n$ в проивольном порядке и поставить вокруг них скобки, обозначив тем самым цикл. Способов сделать это $n!$ по теореме~3.17, однако если при циклических сдвигах этой перестановки мы получаем на самом деле одинаковые перестановки. Например: $$(1234) = (2341) = (3412) = (4123)$$
Всего таких сдвигов $n$ штук. Поделив на эту величину первоначальное количество циклов, которое мы получили, имеем
$$\frac{n!}{n} = (n-1)!$$
\end{proof}

\begin{definition}
\term{Типом перестановки} называется последовательность $(c_1, c_2, \ldots, c_n)$, где $c_k$~--- количество циклов длины $k$ заданной перестановки.
\end{definition}

\begin{example}
Типом перестановки $\pi$, используемой выше, будет $(0, 1, 1, 0, 0)$, типом перестановки $\rho$~--- $(0, 0, 0, 0, 1)$. Тривиальная перестановка $1_{[n]}$, которая оставляет все элементы на месте, имеет тип $(n, 0, 0, 0, 0)$, то есть она состоит из $n$ циклов вида $(a)$, то есть только таких циклов, которые оставляют элемент на месте.
\end{example}

\begin{exercise}
Покажите, что для любого типа $(c_1, c_2, \ldots, c_n)$ всегда выполнено соотношение
$$\sum_{i=1}^n ic_i = n$$
\end{exercise}

\begin{thm}
Существует
$${n!\over c_1!c_2!\ldots c_n!1^{c_1}2^{c_2}\ldots n^{c_n}}$$
перестановок типа $(c_1, c_2, \ldots, c_n)$
\end{thm}