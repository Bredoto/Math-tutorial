\section{О боге}

\begin{thm}
Бог существует.
\end{thm}
\begin{proof}
Это доказательство принадлежит Гёделю, которого мы уже неоднократно поминали в нашем курсе. Некоторые исследователи считают\footnote{<<Reflections on Gödel’s Ontological Argument>>, Christopher G. Small}, что попытка доказать существование Бога было одним из главных движущих стимулов для Гёделя заниматься логикой. Достоверно известно, что Гёдель был ревностным католиком, но, видимо, отлично понимал, что вещи, высказываемые в церкви, совершенно смехотворны с точки зрения логической обоснованности и никакого умного человека не убедят. Поэтому он пытался привести строгое доказательство. Рассуждения на тему доказательства существования Бога у него появились в ранних черновиках и занимался он этим доказательством на протяжении всей жизни. Само доказательство он так и не опубликовал, вероятно, считая его неполноценным и неубедительным, и та форма, в которой доказательство сейчас приводится в многочисленных источниках, восстановлена по его черновикам после смерти и комментариям его студентов, с некоторыми из которых он обсуждал доказательство.

Я думаю, что привести здесь это доказательство будет довольно интересно и познавательно <<для общего развития>>. В конце концов есть подозрения, что современная логика развивалась во многом с целью развития именно этого доказательства, так же оно является неплохим примером применения модальной логики. Сегодня вариации этого доказательства появляются с завидной регулярностью, и когда в новостях иногда проскакивает что-нибудь вроде <<математики доказали существование Бога>>~--- это вовсе не журналистская утка, это появление одного из вариантов подобного доказательства.

Прежде чем мы перейдём к формальному доказательству, я разъясню базовые идеи Гёделя.

Во-первых, Гёдель опирается на идею о том, что любой объект обладает некоторыми \term{свойствами}, которые его целиком описывают. Это довольно простая идея: те же физики, когда получают новую частицу, тщательно описывают её в каждом мельчайшем аспекте. По этому описанию мы всегда можем понять о какой частице идёт речь. Договоримся, что запись $Fx$ будет означать, что объект $x$ обладает свойством $F$. Таким образом $F$ является предикатом.

Определим объединение ($F\lor H$) и пересечение ($F \land H$) свойств следующим образом:
$$\Box \forall x, (F\land H) x \leftrightarrow \Box \forall x, Fx\land Hx$$
$$\Box \forall x, (F\lor H) x \leftrightarrow \Box \forall x, Fx\lor Hx$$


Набор свойств (предикатов) $\textbf{F}$ мы будем обозначать жирным шрифтом. Свойство, являющееся объединением всех свойств набора $\textbf{F}$ будем обозначать как $\bigvee\textbf{F}$, а свойство-пересечение как $\bigwedge\textbf{F}$. Набор всех свойств объекта $x$ обозначим как $\textbf{X}$, а \term{сущностью} этого объекта назовём свойство $X = \bigvee \textbf{X}$.


\end{proof}