<div><strong>Определение.</strong> <em>Функцией</em>, или <em>отображением</em>, из множества $latex A$ во множество $latex B$ (обозначение $latex f:A\to B$) называется множество упорядоченных пар $latex f\subset A\times B$, таких что для любого $latex a\in A$ найдется единственный $latex b\in B$, такой что $latex (a, b) \in f$.</div>
Если $latex (a, b) \in f$, то это часто записывается как $latex f(a) = b$ или как $latex f:a \mapsto b$. Множество всех функций $latex \{f: A\to B\}$ обозначается как $latex B^A$.

<strong>Определение.</strong> Если $latex f:A \to B$, то множество $latex A$ называется <em>областью определения</em> функции $latex f$ и обозначается как $latex \mathrm{Dom} f$.

Область определения может представлять собой декартово произведение нескольких множеств. В этом случае говорят, что функция является <em>функцией нескольких переменных</em>, где каждое множество соответствует отдельной переменной.

<strong>Определение.</strong> Если $latex f:A \to B$, то множество $latex B$ называется <em>областью значений</em> функции $latex f$ и обозначается как $latex \mathrm{Codom} f$.

<strong>Пример.</strong> Логические операции И, ИЛИ, Исключающее ИЛИ, импликация и эквиваленция рассматриваемые нами ранее, являются фунциями $latex f:B\times B \to B$ (это всё функции нескольких переменных), где $latex B = \{0, 1\}$. Функция НЕ является функцией типа $latex f: B \to B$.

<strong>Пример.</strong> Объединение и пересечение множеств являются функциями типа $latex f: S \times S \to S$, где $latex S$ — некоторое множество, состоящее из других множеств.

<strong>Пример.</strong> Любой предикат $latex F$, заданный на некотором множестве $latex S$ является функцией типа $latex F: S \to B$, где $latex B = \{0, 1\}$.

Если говорить о чистой интуиции, то понятие функции имеет две основных трактовки. Первая — это некоторый объект, который по заданному элементу множества $latex A$ каким-то образом выдает какой-то элемент множества $latex B$. Иногда он берет его из таблицы, иногда есть какая-то формула, по которой можно этот самый $latex b$ вычислить, иногда какая-то написанная на компьютере программа, которая по $latex a$ дает $latex b$. Если у нас есть некоторое физическое устройство (например, черный ящик) с клавиатурой, свалившееся из космоса, которое при наборе какого-то числа дает в ответ другое число, и мы не знаем как именно оно это делает — это все равно тоже функция.

Мы можем привести множество примеров функций в быту. Если $latex A$ — множество женатых мужчин, а $latex B$ — замужних женщин, то функция, сопоставляющая каждому мужчине в соответствие его жену, является функцией вида $latex f: A\to B$. Если у нас например есть база данных, в которую мы можем вводить имена мужчин, а она отдает нам в ответ имена их жен, то эта база данных как раз и будет реализовывать данную функцию.

Можно рассмотреть множество рабочих дней, прошедших от начала торгов на Нью-Йоркской фондовой бирже, и каждому дню сопоставить лидеров роста и падения. Если дни обозначить за $latex D$, а компании за $latex C$, то такое сопоставление будет функцией вида $latex D\to C\times C$.

В дальнейшем мы будем временами приводить примеры из кодирования и криптографии. Шифрование и кодирование — это тоже функции. Если $latex T$ — множество всех возможных текстов, а $latex B^\infty$ — множество последовательностей нулей и единиц (как это принято в компьютере на низком уровне), кодирование — это функция $latex f:T\to B^\infty$. Задачей теории кодирования является построение такой функции $latex f$, чтобы она обладала какими-то полезными свойствами, например чтобы запись была максимально краткой, либо чтобы она была устойчива к ошибкам, и в случае каких-то сбоев можно было восстановить весь текст и по фрагменту кодировки.

Если $latex K$ — множество ключей, а $latex C$ — множество шифровок, то шифрование сводится к реализации функции шифрования $latex E: T\times K \to C$ и дешифрования $latex D: C\times K \to T$, которые должны быть выбраны таким образом, чтобы не зная ключа $latex k\in K$ нельзя было восстановить исходный текст $latex t \in T$ по шифру $latex c \in C$.

Если $latex A$ — множество карточных мастей, а $latex B$ — множество достоинств карт, то функция, которая ставит каждой карте в соответствие ее достоинство, имеет вид $latex f: A\times B \to B$ и может быть записана формулой как $latex f: (a, b) \mapsto b$.

Второй интуитивный смысл, который часто имеют функции — это установление соответствия между различными объектами, которое говорит нам что-либо о свойствах этих объектов. Пусть, например, множество $latex A$ состоит из упорядоченных элементов $latex a&lt;b&lt;c$, а множество $latex B$ из элементов $latex a&lt;b&lt;c&lt;d&lt;e$. Тогда функция $latex f:A \to B$, которая сопоставляет любому элементу тот же самый элемент другого множества ($latex f: x\mapsto x$) может показать нам, что множество $latex A$ в некотором смысле является начальным отрезком множества $latex B$ — любой элемент последнего множества, не нашедшего себе пары во множестве $latex A$, будет больше любого другого элемента. Это примитивный пример, но вероятно он как-то продемострирует общую идею (если нет, то позже вероятно вы это поймете на практике).

Рассмотрим более содержательный пример. В первом параграфе мы говорили, что любому предикату на множестве $latex S$ соответствует подмножество $latex S$ и наоборот. Это соответствие — тоже функция. Каждый такой предикат — это функция вида $latex f:S\to B$, и стало быть $latex f\in B^S$. Тогда функция $latex p$, которая по предикату дает подмножество, ему соответствующее, имеет тип $latex p:B^S \to 2^S$. Забегая вперед можно сказать, что $latex 2=\{0, 1\}$ (это будет объяснено в третьей главе), и именно отсюда происходит обозначение для булеана.

Отметим так же, что функции используются часто не только для отображения отдельных элементов, но и для отображения подмножеств элементов.

<strong>Пример.</strong> Пусть опять $latex A$ — множество женатых мужчин, $latex B$ — замужних женщин, а $latex f: A\to B$ ставит каждому мужчине в соответствие его жену. Пусть теперь $latex M\subset A$ — подмножество мужчин, работающих в Макдональдсе. Тогда $latex f(M)$ — это множество женщин, чьи мужья работают в МакДональдсе.

Это можно формализовать при желании и назвать отдельными словами:

<strong>Определение.</strong> Если $latex f(a) = b$, то $latex b$ называется <em>образом</em> элемента $latex a$ по отображению $latex f$.

<strong>Определение.</strong> Множество $latex f(S) = \{y\in \mathrm{Codom}f|\exists x \in \mathrm{Dom}f, f(x) = y \}$ называется <em>образом</em> множества $latex S$ по отображению $latex f$.

<strong>Определение.</strong> Множество $latex f^{-1}(y) = \{x | f(x) = y \}$ называется <em>прообразом</em> элемента $latex y$.

<strong>Определение.</strong> Множество $latex f^{-1}(S) = \{x | f(x) \in S \}$ называется <em>прообразом</em> множества $latex S$.

Обратите внимание на то, что образом любого элемента является только один элемент, а прообразом является целое множество элементов (вполне возможно, что пустое).

<strong>Пример.</strong> Пусть $latex A = \{a, b, c\}$, $latex f = \{(a, a), (b, c), (c, c)\}$. Тогда $latex f^{-1}(a) = \{a\}$, $latex f^{-1}(b) = \emptyset$, $latex f^{-1}(c) = \{b, c\}$.

<strong>Определение.</strong> Множество $latex \mathrm{Im} f = f(\mathrm{Dom} f)$ называется <em>образом</em> функции $latex f$.

Обратите внимание на то, что в общем случае $latex \mathrm{Im} f \not= \mathrm{Codom} f$. Так, в последнем примере $latex \mathrm{Im} f = \{a, c\}$, но $latex \mathrm{Codom}f = \{a, b, c\}$.

<strong>Определение.</strong> Единичной функцией на множестве $latex A$ называется функция $latex 1_A: A\to A$, ставящая любому элементу в соответствие его же самого: $latex 1_A: a\mapsto a$.

<strong>Определение.</strong> <em>Композицией</em> функций $latex f:B\to C$ и $latex g:A\to B$ называется функция $latex f\circ g: A\to C$, такая что если $latex f(b) = c$ и $latex g(a) = b$, то $latex (f\circ g)(a) = c$.

<strong>Теорема.</strong> Для любой функции $latex f: A\to B$, $latex 1_B \circ f = f \circ 1_A = f$.

Докзательство в качестве простого упражнения.

<strong>Теорема.</strong> Композиция функций ассоциативна: $latex f\circ (g \circ h) = (f\circ g)\circ h$.

<strong>Доказательство.</strong> Достаточно выписать напрямую два значения функции для произвольного элемента $latex x$, чтобы увидеть это:

Слева: $latex (f\circ (g \circ h)) (x) = f((g\circ h)(x)) = f(g(h(x)))$

Справа: $latex ((f\circ g) \circ h) (x) = (f\circ g)(h(x)) = f(g(h(x)))$

Как видно, в обоих случаях получается одно и то же значение. $latex \blacksquare$

<strong>Определение.</strong> Функция называется <em>инъективной</em>, или <em>инъекцией</em>, если $latex f(a)\not= f(b)$ для любых $latex a\not= b$.

<strong>Определение.</strong> Пусть $latex f:A\to B$. Функция $latex f^{-1}_l$, такая что $latex f^{-1}_l\circ f = 1_A$ называется <em>левой обратной</em>.

<strong>Теорема.</strong> Функция имеет левую обратную функцию тогда и только тогда, когда она инъективна.

Докажите эту теорему в качестве упражнения.

<strong>Пример.</strong> Любая функция кодирования обязана быть инъективной, поскольку в противном случае была бы возможна ситуация $latex f(a) = f(b) = c$, и было бы не понятно как мы должны раскодировать $latex c$ обратно.

<strong>Определение.</strong> Если $latex \mathrm{Im}f = \mathrm{Codom}f$, то функция называется <em>сюръективной</em>, или <em>сюръекцией.</em>

<strong>Определение.</strong> Пусть $latex f: A\to B$. Функция $latex f^{-1}_r$. такая что $latex f\circ f^{-1}_r = 1_B$ называется <em>правой обратной</em>.

<strong>Теорема.</strong> Функция имеет правую обратную функцию тогда и только тогда, тогда она сюръективна.

Доказательство опять же не сложно и я осталвяю его читателю в качестве упражнения.

<strong>Упражнение.</strong> Пусть $latex f:A \times B \to A$ и $latex f: (a, b)\mapsto a$. Докажите, что эта функция сюръективна.

<strong>Определение.</strong> Если функция одновременно и сюръективна и инъективна, то она называется <em>биективной</em>, либо <em>биекцией</em>.

<strong>Теорема.</strong> Если $latex f: A\to B$ — биекция, то левая обратная функция будет совпадать с правой обратной функцией.

Доказательство снова в качестве не сложного упражнения. Понятно, что в случае с биекциями разница между левой обратной и правой обратной функцией пропадает (в случае же сюръекции или инъекции существует лишь одна из них), и поэтому такая функция называется просто <em>обратной</em>.

Так же полезно заметить, что произвольную инъективную функцию возможно сделать биективной, если заменить ее область значений лишь ее образом, то есть если $latex C = \mathrm{Im} f$, то вместо функции $latex f: A\to B$ рассмотреть функцию $latex f: A \to C$. Легко проверить, что в этом случае функция действительно станет биекцией.

Для сюръекции подобное утверждение тоже верно, но только при отдельных оговорках.

<strong>Определение.</strong> Ограничением функции $latex f: A\to B$ на $latex S\subset A$ называется функция $latex f|_S: S\to B$, такая что для любого $latex x\in S$ верно, что $latex f|_S(x) = f(x)$.

Несколько более формально и точно, но менее понятно можно написать, что $latex f|_S = f \cap S \times B$.

Для произвольной сюръективной функции можно было бы попробовать искать такое ограничение этой функции, чтобы она стала биекцией. Предположение это на первый взгляд довольно очевидно, однако оказывается, что оно эквивалентно так называемой аксиоме выбора, которую во-первых нельзя взять и доказать, а во-вторых из которой следует множество парадоксов. Подробнее мы будем обсуждать эту тему далее в этом курсе (и совсем немного в следующем параграфе), пока что можно просто принять к сведению (хотя это и не принципиальной важности факт), что доказать существование такого ограничение невозможно.

<strong>Определение.</strong> Множества $latex A$ и $latex B$ называются <em>равномощными</em> (обозначение $latex |A| = |B|$), если существует биекция $latex f: A\to B$.

Равномощность говорит о том, что элементы множеств $latex A$ и $latex B$ можно поставить во взаимооднозначное соответствие. Часто это интерпретируется как то, что они содержат одинаковое число элементов. Это правда довольно опасная интерпретация, что станет ясно, когда мы начнем говорить о бесконечных множествах. Пока же в принципе довольно удобно воспринимать равноможность именно так. имея  правда ввиду, что это сгодится лишь только для довольно маленьких множеств.

<strong>Пример.</strong> Пусть $latex A = \{1, 2, 3\}$, $latex B = \{a, b, c\}$. Тогда эти множества равномощны, поскольку существует биекция $latex f=\{(1, a), (2, b) , (3, c)\}$.

<strong>Пример.</strong> Пусть $latex A$ — множество женатых мужчин, а $latex B$ — множество замужних женщин. Эти множества равномощны.

<strong>Упражнение.</strong> Приведите пример неравномощных множеств и двух отображений на них: сюръективного и инъективного.
