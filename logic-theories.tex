\section{Теории и выводимость}

Этот и следующий параграфы делают попытку довольно неформально и на пальцах объяснить вещи, которые обычно изучаются лишь на специализированных курсах университетов и очень в формализованном виде. Если вы ничего здесь не поймете, материал возможно станет понятнее после прочтения второй главы.

{\bfseries Определение.} {\slshape Правилом вывода}, или {\slshape трансформации формул,} будем называть правило, которое  множеству формул ставит в соответствие новую формулу.

{\bfseries Определение.} Правило вывода называется {\slshape корректным}, если при условии истинности всех начальных формул, будет истинной и формула, которая следует по правилу вывода.

Это очень важная характеристика правил вывода, поскольку она гарантирует, что из формул, которые при определенных значениях переменных являются истинными, мы всегда будем выводить истинные формулы при тех же значениях переменных.

Записывать правила вывода мы будем как список начальных формул, по одной формуле в каждой строке, и отделенных подчеркиванием от результата применения правила.

Вот вам для начала несколько примеров корректных правил трансформации:

1) Сокращение двойного отрицания:

$\begin{array}{l} \neg\neg\phi\\ \hline \phi\end{array}$

2) Введение двойного отрицания:

$\begin{array}{l}\phi\\ \hline\neg\neg \phi\end{array}$

3) Введение конъюнкции

$\begin{array}{l}\phi \\ \psi\\ \hline \phi\wedge \psi\end{array}$

4) Сокращение конъюнкции

$\begin{array}{l}\phi \wedge \psi\\ \hline \phi\end{array}$

5) Введение дизъюнкции

$\begin{array}{l}\phi \\ \hline \phi \vee \psi\end{array}$

6) Дизъюнктивный силлогизм

$\begin{array}{l}\phi \vee \psi\\ \neg\phi \\ \hline \psi\end{array}$

7) Сокращение эквиваленции:

а) $\begin{array}{l}\phi \leftrightarrow \psi\\ \phi \\ \hline \psi\end{array}$

б) $\begin{array}{l}\phi \leftrightarrow \psi\\ \neg\phi \\ \hline \neg\psi\end{array}$

в) $\begin{array}{l}\phi \leftrightarrow \psi\\ \phi \vee \psi \\ \hline \phi \wedge \psi\end{array}$

г) $\begin{array}{l}\phi \leftrightarrow \psi\\ \neg\phi \vee \neg\psi \\ \hline \neg\phi \wedge \neg\psi\end{array}$

Я приводил только односторонние правила. Например, при сокращении эквиваленции, я указывал только на сокращение с одной стороны. Можно привести правило и для сокращения с другой стороны.

{\bfseries Упражнение.} Докажите, что приведенные правила вывода действительно корректны.<strong>
</strong>

{\bfseries Определение.} {\slshape Теорией} мы будем называть такое множество формул, что при применении к любым формулам этой теории правил трансформации, мы всегда будем получать формулы этой же теории.

{\bfseries Определение.} Теория называется {\slshape аксиоматизируемой}, если возможно выделить конечный набор формул, называемых {\slshape аксиомами}, такой, что все остальные формулы теории возможно получить, применяя последовательность правил вывода к аксиомам.

Эти определения могут показаться странными и непонятными. Интуитивно об этом проще думать так: в начале задается некоторый набор аксиом, а уже из них пользуясь правилами вывода выводятся различные формулы (они так же называются {\slshape теоремами}). Объединение всех аксиом со всеми теоремами, из них выводимыми (это бесконечное множество), как раз и называется аксиоматизируемой теорией (неаксиоматизируемые теории нам в этом курсе не интересны).

Существенно так же заметить, что если теория аксиоматизируема, то сам набор аксиом для нее не определен однозначно — одна и та же теория может аксиоматизироваться разными способами.

Факт принадлежности формулы к теории записывается как $T\vdash\phi$ (это по сути то же самое, что и $\phi \in T$). При этом говорят, что формула $\phi$ {\slshape выводима, }или является{\slshape  синтаксическим следствием,} в теории $T$. Часто так же отдельно оговаривают, каким множеством правил вывода пользуется теория. Логично, что от набора правил вывода состав теории очень сильно зависим. Мы не будем себя ограничивать и будем говорить, что у нас во всех теориях используются все правила вывода, упомянутые в курсе.

Удобно считать, что для любой тавтологии определено правило вывода из пустого набора формул. То есть, например, если взять тавтологию $a\oplus b \leftrightarrow (a\vee b) \wedge (\neg a \vee \neg b)$, то можно написать следующее «правило вывода»:

$\vdash \phi\oplus \psi \leftrightarrow (\phi\vee \psi) \wedge (\neg \phi \vee \neg \psi)$

Отсутствие указания на теорию слева от знака выводимости говорит о том, что данная формула содержится в любой теории. (Таким образом возможно построение аксиоматизируемой теории с пустым набором аксиом — такая теория будет содержать все логические тавтологии).

В качестве примера всего понаписанного, давайте докажем, что если $T\vdash \phi\oplus\psi$, то тогда $T,\phi\vdash\neg\psi$ (запись слева от знака выводимости говорит о том, что мы рассматриваем теорию $T$, дополненную формулой $\phi$). Причем докажем это не по таблице истинности, а именно с точки зрения синтаксической выводимости.

1) $T \vdash \phi \oplus \psi$ — это наше условие.

2) $T \vdash \phi\oplus \psi \leftrightarrow (\phi\vee \psi) \wedge (\neg \phi \vee \neg \psi)$ - тавтология, которую мы намереваемся использовать.

3) $T\vdash (\phi\vee \psi) \wedge (\neg \phi \vee \neg \psi)$ — исходя из 1) и 2) по правилу о сокращении эквиваленции.

4) $T\vdash \neg \phi \vee \neg \psi$ — исходя из 3) по правилу о сокращении конъюнкции.

5) $T, \phi \vdash \neg \phi \vee \neg \psi$ — если добавить к теории формулу, то все формулы, выводимые ранее, останутся выводимы.

6) $T, \phi \vdash \neg \psi$ —  по правилу о дизъюнктивном силлогизме.

Что и требовалось доказать.

{\bfseries Упражнение.} Попробуйте доказать $T\vdash \phi\oplus\psi$ исходя из двух условий $T, \phi \vdash \neg \psi$ и $T, \psi \vdash \neg \phi$. Возможно ли вывести то же самое, отказавшись от одного из условий?

{\bfseries Упражнение.} {\slshape Прошлое упражнение было удалено отсюда из-за неустранимой ошибки в условии. Скоро вставлю сюда что-нибудь новое.}

В качестве еще одного примера докажем довольно концептуальную штуку. Пусть наша теория такая, что одновременно $T\vdash\phi$ и $T\vdash\neg\phi$ (то есть из одной теории выводится одновременно и некоторая формула $\phi$, и ее отрицание). Тогда можно провести следующую цепочку рассуждений:

1) $T\vdash\phi$ — по условию.

2) $T\vdash\phi\vee\psi$ — введение дизъюнкции. Здесь $\psi$ — произвольная формула.

3) $T\vdash\neg\phi$ — по условию.

4) $T\vdash\psi$ — исходя из 2) и 3) и дизъюнктивного силлогизма.

Таким образом получилось, что из теории $T$ мы можем вывести совершенно произвольную формулу. Такие теории естественно имеют мало смысла, и называются {\slshape противоречивыми}. Характеризуются они выводимостью из них одновременно какой-либо формулы и ее отрицания.

Если теория $T$, с которой мы работаем, непротиворечива, то из сказанного можно заключить, что корректно так же следующее правило вывода:

$\begin{array}{l} T,\phi\vdash\psi\\ T,\phi\vdash\neg\psi\\ \hline T\vdash\neg\phi\end{array}$

Следующие правила вывода потребуют некоторого напряжения:

{\bfseries Universal Instantiation (UI):} $\begin{array}{l}\forall x\phi(x)\\ \hline\phi(\alpha)\end{array}$

{\bfseries Existential Instantiation (EI):} $\begin{array}{l}\exists x\phi(x)\\ \hline\phi(\alpha)\end{array}$

{\bfseries Existantial Generalization (EG):} $\begin{array}{l}\phi(\alpha)\\ \hline\exists x\phi(x)\end{array}$

{\bfseries Universal Generalization (UG)}: $\begin{array}{l}\phi(\alpha)\\ \hline\forall x\phi(x)\end{array}$

Эти правила работают далеко не всегда. Чтобы их применять, необходимо быть уверенным в том, что имена переменных, которые вы используете в ваших манипуляциях с кванторами и переименованиями, нигде в теории больше не задействованы, так что ваши манипуляции не повлияют на остальную теорию. Я для начала покажу пример вывода с использованием этих правил, а затем мы немного детальнее обсудим откуда они берутся и как интерпретируются.

Вот как можно решить упражнение из прошлого параграфа:

1) $T\vdash\forall x, (P(x)\wedge Q(x))$ — начальное условие.

2) $T\vdash P(a)\wedge Q(a)$ — применение UI

3) $T\vdash P(a)$ — сокращение конъюнкции

4) $T\vdash\forall x, P(x)$ — применение UG

5) $T\vdash Q(a)$ — еще раз сокращение конъюнкции для 2), но уже с другой стороны.

6) $T\vdash \forall x, Q(x)$ — применение UG

7) $T\vdash (\forall x, P(x)) \wedge (\forall x, Q(x))$ — введение конъюнкции для 4) и 6)

Это доказательство в одну сторону. В другую стороны оно выполняется точно так же, но в обратном порядке и с другими названиями для правил. Собственно что и требовалось доказать.

{\bfseries Упражнение.} Докажите с помощью правил вывода для кванторов остальные формулы упражнений прошлого параграфа.

Теперь давайте рассмотрим пример, когда применять эти правила нельзя:

1) $T\vdash \exists x\exists y, x \not= y$ — пусть это наше предположение

2) $T\vdash a \not= b$ — результат двойного применения EI

3) $T\vdash \forall b, b \not= b$ — результат применения UG к $a$.

Последний вывод явно не верен, впрочем, и ошибка наша очевидна: мы переименовали $a$ в переменную, которая уже задействована, чем внесли разлад в теорию. Однако если выбрать другое имя, то получится не лучше:

3) $T\vdash \forall z, z \not= b$ — вторая попытка с другим именем

Тоже ничего хорошего — наше изначальное предположение было в том, что просто существуют элементы, не равные друг другу, а не то что все они не равны одному конкретному элементу (последнее вообще делает нашу теорию противоречивой). В данном случае мы опять же не имели права применять операцию UG к $a$ по той причине, что эта переменная не совсем независимая — изначально при переименовании мы подразумевали, что новое введенное имя $a$ — это имя некоторой переменной, которая не равна $b$. И это предположение об $a$ никуда не делось — следовательно мы не имели права применять UG.

Первый пример отличается от этого тем, что там переменная $a$ изначально имела квантор всеобщности, и применяя операцию UI мы не накладывали на нее никаких ограничений, а следовательно могли его затем обратно смело возвращать. На практике в основном UI и UG так и используются — вначале убирается квантор всеобщности, затем над формулой проводятся какие-то манипуляции (с квантором мы бы их не могли сделать), и в конце квантор возвращается.

Второй способ применить UG не рискуя внести противоречия в теорию — это применить его к формуле, которая изначально появилась в теории из тавтологии. Вводя тавтологии мы можем задавать произвольные имена переменным, и они ни от чего не будут зависеть.

Почему эти правила вообще работают? UI, EI, EG, думаю, особых сомнений не вызывают. Например, UI применялся еще философами Древней Греции, откуда нам досталось классическое рассуждение: «Все люди смертны. Сократ — человек. Следовательно, Сократ смертен». Если записать это на языке логики, то это в точности пример применения правила UI. Для остальных правил так же можно придумать подобную интуицию.

Сомнения может вызывать разве что правило UG. На первый взгляд оно кажется очень странным: из того, что $P(a)$ верно для одного какого-то конкретного $a$, оно оказывается верным сразу для всех $a$. «Из того, что я дурак, следует, что и все дураки» — явно не верное рассуждение. Однако надо помнить, что на применение UG накладывается жесткое условие, чтобы сам объект утверждения, к которому мы применяем правило, был независим от остальных теорем. А раз в независимости от остальных теорем у нас оказался выводим $P(a)$, то и для любого другого объекта мы тоже могли это вывести. На практике это либо возврат ранее снятого квантора, либо введение тавтологии.

{\bfseries Упражнение.} Развлеките сами себя. Придумайте какую-нибудь содержательную теорию.
