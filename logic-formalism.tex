\section{Формализм}

До сих пор все наши рассуждения были главным образом интуитивными, мы аппелировали к каким-то физическим образам и вводили нестрогие вспомогательные понятия вроде множеств (само понятие множества определяется в математике строго, но мы это сделаем лишь в следующей главе). Такой подход нельзя назвать безупречным с математической точки зрения, поэтому в этой главе мы формально введём уже рассматриваемые нами ранее понятия.  Вернее, не совсем формально, но я покажу, как это в целом делается\footnote{Если хотите максимально формального изложения, то смотрите, например, Бурбаки <<Теория множеств>>}. Этот параграф необязателен для дальнейшего понимания книги, но тем, кому важно доскональное понимание основ, он сможет немного помочь.

Состоять наше изложение логики будет из трёх частей: языка (то, как мы записываем предложения), синтаксиса (правила вывода одного предложения из другого) и семантики (наделение предложений предполагаемым смыслом).

Как мы уже отмечали, когда мы даём какое-то определение, мы всегда вынуждены пользоваться другими определениями. В конечном итоге мы обязаны ввести какое-то понятие, которое мы никак не определяем. Это называется <<принципом Мюнгхаузена>>: если бы мы не начинали построение математики от какого-то неопределяемого понятия, то получилось бы, что наши определения как-то зависимы друг от друга: условно говоря определение А базировалось бы на определении Б, а определение Б на определении А, и это в самом явном случае (зависимости могли бы быть самыми сложными теоретическими, но такие определения всегда были бы ошибочны).

Ладно, на самом деле, если быть совсем уж дотошными, мы всё же можем вводить определия, ссылающиеся сами на себя (такие определения называются рекурсивными), но с некоторыми оговорками. Например, в математике и иноформатике часто встречается понятие дерева, которое определяется так: дерево~--- это набор (возможно, пустой) деревьев. Как иллюстрацию можно рассмотреть генеалогическое дерево. Генеалогическое дерево человека можно рассматривать как набор из двух деревьев: генеалогическое дерево отца и генеалогическое дерево матери. Их деревья так же можно рассматривать как наборы из двух деревьев и т.д. Если предки человека нам не известны, то его генеалогическое дерево будет состоять из пуского набора деревьев.

Деревья, в том числе и бесконечные, часто имеют важный инженерный смысл: если, скажем, взять некоторую игру, то возможные действия игрока можно рассматривать как дерево возможностей. Каждое действие игрока в этом случае можно рассматривать как выбор одного дерева из данного набора. Затем он производит некое действие, что опять же соответствует новому дерему, содержащихся в данном дереве. Анализом подобных деревьев занимается теория игр, и техники, разработанные этой наукой, частенько встречаются в системах искусственного интеллекта (как вариант в тех же компьютерных играх).

Впрочем, в понятии <<дерева>> мы так же использовали понятие <<набор>>. Его тоже надо как-то определить. Мы могли бы сказать, что набор~--- это какой-то объект составленный вместе с ещё каким-то набором. В этом случае надо определить что такое <<объект>> и что такое <<составленный вместе>>. Причем если мы хотим рассматривать конечные наборы, то надо отдельно оговорить, что набор может быть пустым, то есть не состоящим из других наборов, и само это понятие так же требует определения.

Из этих рассуждений видно, что мы в общем-то в любом случае должны смириться с тем, что какие-то вещи у нас останутся неопределенными, даже не смотря на то, что нам доступна рекурсия. В качестве понятия, которое мы никак не будем определять, у нас будет выступать \term{<<символ>>}. С точки зрения интуиции, символ~--- это некоторая закорючка на бумаги. С точки же зрения логики это понятие, которое мы принимаем без попыток понять что это.

\begin{definition}
\term{Алфавитом} назовём набор символов.
\end{definition}

\begin{example}
Примером алфавитов может служить русский или английский алфавит.
\end{example}

\begin{example}
Для нужд логики мы определим алфавит, состоящий из символов $\land$, $\lor$, $\to$, $\neg$, $\oplus$, $\leftrightarrow$, $\forall$, $\exists$, $=$, (, ) а так же всех символов английского языка, как строчных, так и заглавных. Сам этот алфавит будем обозначать как $\Sigma$.
\end{example}

Да, опять же нам надо определить что такое набор, и это возможно сделать, определив <<набор>> как выше, заменив <<объект>> <<символом>>, а так же введя неопределяемые понятия <<и>> и <<составлены вместе>>. Но в такую степень формализма мы уже не будем углубляться, и если вы в дальнейшем найдёте где-то подобные дырки в определениях, отнеситесь к этому с пониманием.

\begin{definition}
\term{Строкой} (или так же \term{словом}) называется конечная упорядоченная последовательность сиволов (возможно, пустая) некоторого алвавита. Пустая строка для удобства обозначается как $\epsilon$.
\end{definition}

Строками алвавита $\Sigma$, определённого выше, являются например такие выражения как <<$\land\land P\oplus\leftrightarrow$>>. Строками русского алфавита будут такие последовательности как <<аплотфдц>>, а английского такие как <<shehs>>. Это совершенно бессмысленный набор символов, и отсюда ясно, что нам необходимо как-то из всех возможных строк выделить допустимые.

\begin{definition}
Набор слов (возможно, бесконечный) называется \term{языком}.
\end{definition}

Простейший, но одновременно с тем и почти бесполезный способ задания языка~--- это простое перечисление всех строк алфавита. В каких-то частных случаях это было бы возможно, но в целом это неинтересные либо непрактичные примеры. Поэтому перечислить все возможные предложения логики (или любого другого языка) без использования каких-то специальных механизмов явно невозможно.

\begin{exercise}
\term{Грамматикой} называется некоторое формальное описание структуры допустимых слов языка.
\end{exercise}

Это довольно нечёткое определение. И как именно грамматику задавать может решать каждый сам для себя. Например, мы могли бы сконструировать язык, описывающий все возможные положения игры крестики-нолики. В качестве алфавита мы выбрали бы набор $\{x, o, ?,1, 2\}$, а в качестве языка условились бы называть все строки длины 10 этого алфавита, в которых первым символом идёт либо 1 либо 2 (что обозначает игрока, которому принадлежит ход), а оставшиеся символы обозначали бы подряд все клетки поля, где помимо крестиков и ноликов мы могли бы ставить символ $?$ для незанятых клеток. Примером такого слова может служить строка <<20?0?x???xx>>~--- она описывает ситуацию, изображенную в таблице~1.11.

\begin{table}[h]
\centering
\begin{tabular}{c | c | c}
o & & o\\
\hline
  & x & \\
\hline
 & x & x
\end{tabular}
\caption{Ход второго игрока}
\end{table}

Мы могли бы перечислить все слова языка крестиков-ноликов, но это было бы сложно, так как возможных слов, которые нам подходят, слишком много. Нам оказалось проще описать неформально струкруту языка, и это в общем-то был полноценный пример грамматики. Однако, для задания грамматик и работы с ними существует целый ряд стандартных механизмов, которыми занимается раздел математики под названием <<Теория формальных языков>>, и этими механизмами удобно пользоваться.

Самый простой способ задания грамматики языка~--- это разбить предложения языка на составные единицы и указать правила, по которым они составляются. Тут можно вспомнить уроки русского языка в школах: все изучали, что предложение состоит из составных частей вроде подлежащего, сказуемого, дополнения, вводного предложения, деепричастного оборота. Определения частей языка могут быть и рекурсивными, так, предложением является так же и набор из нескольких предложений, соединённых союзами. Эти элементы языка так же описываются с помощью других языковых элементов: подлежащее может быть представлено местоимением, существительным, числительным и т.п.

Таким же путём мы сейчас опишем грамматику предложений логики. Для удобства мы будем записывать правила в форме, подобной примеру

$$a \to bad | ef | g$$

Здесь $a$~--- это объект языка, вертикальными чертами разделены различные варианты чем $a$ может являться, а между вертикальных черт записывается конкретная форма. Запись может разбиваться на несколько строк. Мы так же для обозначения пустой строки введём символ $\epsilon$.

\begin{example}
Правило, приведённое выше говорит, что запись  $bbefdd$ является элементом $a$. Действительно, $ef$ является элементом $a$. То есть мы можем сказать, что $bbefdd = bbadd$. Теперь, $bad$ в середине строки так же является элементом $a$: $bbadd = bad$. Собственно мы получили, что изначальная строка является элементом $a$.
\end{example}

Я не буду выписывать грамматику целиком, определив лишь базовые конструкиции:\\
\\
Формула $\to$ атом | $\neg$ формула\\
\hspace*{2cm}| формула $\land$ формула | формула $\lor$ формула\\
\hspace*{2cm}| формула $\to$ формула | формула $\leftrightarrow$ формула\\
\hspace*{2cm}| $\forall$ \term{переменная} формула | $\exists$ \term{переменная} формула\\
Атом $\to$ терм = терм | \term{предикат} (списоктермов)\\
Терм $\to$ \term{константа} | \term{операция} (списоктермов)\\
Списоктермов $\to$ терм списоктермов | $\epsilon$\\
\\
Слова, которые я выделил курсивом (предикат, переменная, операция, константа)~--- это некоторые символы, которые мы каким-то произвольным образом разбили в группы. В каждой конкретной ситуации вы разбиваем эти символы по-разному. Например, так:\\
\\
Предикат $\to$ P | Q | R| $\ldots$\\
Переменная $\to$ x | y | z| $\ldots$\\
Операция $\to$ f | g | h | $\ldots$\\
Константа $\to$ a | b | c | $\ldots$\\
\\
Это уже целиком зависит от того, как мы собираемся использовать эти символы.

\begin{example}
Давайте разберём формулу $\forall x P(x) \to Q(f(y))$. Эта формула явно имеет вид <<$\forall$ переменная формула>>, в роли переменной выступает $x$, а в роли формулы $P(x) \to Q(f(y))$. Последняя так же состоит из двух формул $P(x)$ и $Q(f(y))$, соединённых символом $\to$. Обе эти формулы являются атомами вида <<предикат (списоктермов)>>. В случае $P(x)$ список термов состоит из единственного терма $x$ (более точно~--- из терма $x$ и спискатермов $\epsilon$), а в случае $Q(f(y))$ из единсвтенного терма $f(y)$. $x$ является переменной, а терм $f(y)$ имеет вид <<операция список термов>>. В качестве операции тут выступает $f$, а в качестве терма $y$, который является переменной.
\end{example}

Как видно, простая на вид запись на самом деле, если определять её формально, имеет довольно сложную структуру. Тем не менее для математической строгости мы обязаны это всё определять именно таким образом.