\section{Сочетания}

<<$k$-cочетание>>~--- это просто другое название для термина <<$k$-элементное подмножество>>, которое по историческим причинам принято в комбинаторике. Нас будет интересовать количество $k$-соченатий взятых из множества $[n]$.

\begin{definition}
\term{Биномиальным коэффициентом} $n \choose k$ называется число сочетаний из $[n]$ по $k$.
\end{definition}

Вместо обозначения ${n \choose k}$ в российской и французской литературе исторически чаще используется обозначение $C^k_n$, однако оно кажется мне менее удобным, в случае если вместо величин $n$ и $k$ используются какие-то длинные выражения. По этой причине мы будем придерживаться общемирового обозначения.

Значения ${n \choose 1} = n$ и ${n \choose n} = 1$ очевидны. Так же удобно полагать, что ${n \choose 0} =1$ (пустое множество всего одно, соответственно есть лишь один способ его выбрать). 

\begin{thm}
$$\sum_{k=0}^n{n \choose k} = 2^n$$
\end{thm}
\begin{proof}
В левой части этого выражения строит общее количетсов всех подмножеств множества $[n]$. В правой части на самом деле написано то же самое: $2^n$ есть мощность булеана, как вы видели в~\S~3.1.
\end{proof}

\begin{thm}
$${n \choose k} = {n \choose n-k}$$
\end{thm}
\begin{proof}
Выбрать $k$ элементнов из множества $[n]$ это всё равно что выбрать $n-k$ элементов, которые мы оставим в множестве.
\end{proof}

\begin{thm}
$${n \choose k} = {n-1 \choose k-1} + {n - 1 \choose k}$$
\end{thm}

\begin{proof}
Рассмотрим все $k$-сочетания из множества $[n]$. Сам элемент $n$ может либо принадлежать выбранному подмножеству, либо не принадлежать. В первом случае количетство сочетаний будет равно $n-1\choose k$, во втором случае, поскольку один элемент сочетания уже фиксирован, это будет величина $n-1\choose k - 1$.
\end{proof}

Последняя теорема как в случае чисел Стирлинга позволяет вычислять биномиальные коэффициенты. Однако, для сочетаний мы можем записать и явную формулу, правда, проведя некоторую начальную подготовку.

\begin{definition}
\term{$k$-размещением} мы назовём некоторый упорядоченный набор, состоящий из некоторых $k$ элементов множества $[n]$. Количество размещений из $n$ по $k$ будем обозначать как $n^{\lfloor k\rfloor}$.
\end{definition}

\begin{thm}
$$n^{\lfloor k \rfloor} = \frac{n!}{(n-k)!}$$
\end{thm}
\begin{proof}
Доказательство практически дублирует доказателсьтво количетсва перестановок. На первую позицию мы можем поставить один из $n$ элементов. На вторую позицию один из оставшихся $n-1$ элементов. На третью~---один из $n-2$ элементов. Однако в отличии от перестановок, над надо разместить лишь $k$ элементов, а не все $n$, поэтому мы этот процесс оборвём на $k$-том шаге. В итоге получаем выражение
$$n^{\lfloor k \rfloor} = n (n-1)  (n-2) \ldots (n-k+1)$$
Если это выражение умножить и разделить на $(n-k)!$, получим утверждение теоремы.
\end{proof}

Само обозначение $n^{\lfloor k \rfloor}$ на самом деле почти всегда используется просто для обозначения произведения $k$ подряд убывающих чисел. Эту величину часто называют \term{убывающим факториалом}. В полной аналогии вводится и возрастающий факториал:
$$n^{\lceil k \rceil} = n(n+1)(n+2)\ldots(n+k-1)$$

\begin{exercise}
Покажите, что $n^{\lfloor k \rfloor} = (n-k+1)^{\lceil k \rceil}$
\end{exercise}

\begin{thm}
$${n \choose k} = \frac{n!}{k!(n-k)!}$$
\end{thm}
\begin{proof}
$k$-расстановку мы можем получить, вначале выбрав $k$-элементное подмножество $[n]$, а затем упорядочив его. Способов выбрать такое подмножество есть $n\choose k$ способов, способов упорядочить $k!$. Таким образом получаем соотношение
$$n^{\lfloor k \rfloor} = k!{n\choose k}$$
Поделив обе части на $k!$ и подставив выражение для $n^{\lfloor k \rfloor}$, получаем утверждение теоремы.
\end{proof}

\begin{exercise}
Теоремы 3.22 и 3.23 можно доказать, пользуясь явным представлением биномиального коэффициента, полученным в 3.25. Сделайте это.
\end{exercise}

\begin{thm}
$$(x+y)^n = \sum_{k=0}^n {n \choose k} x^{n-k} y^k$$
\end{thm}
\begin{proof}
Давайте вначале для наглядности распишем степень подробно:
$$(x+y)(x+y)\ldots(x+y)$$
Раскроем все скобки (если пока не понятно как, то потренируйтесь на каких-то частных случаях типа $n=3$). Раскрытие скобок можно интерпретировать так, что из каждой скобки мы выбираем либо $x$, либо $y$. Все слагаемые в полученной сумме будут иметь вид $x^iy^j$, $i+j=n$ с каким-то коэффициентом, появляющимся за счёт того, что некоторые слагаемые вида $x^iy^j$ появляются несколько раз.

Слагаемое $x^n$ появится лишь один раз в случае, если из всех скобок мы выберем $x$. Слагаемое $x^{n-1}y$ появляется, если мы выбираем $x$ из всех скобок, кроме одной. Эту одну скобку, из которой мы выбираем $y$, мы можем выбрать одним из $n$ способов. По аналогии $x^{n-2}y^2$ появится $n\choose 2$ раз, поскольку мы выбираем уже две скобки, из которых мы возьмём $y$. Продолжая по аналогии приходим к утверждению теоремы.
\end{proof}

\begin{exercise}
Приведённую теорему можно доказать по индукции. Будет полезным проделать это самостоятельно.
\end{exercise}

Приведённая теорема даёт нам ещё один способ подсчитать количество подмножеств множества $[n]$:

$$\sum_{k=0}^n {n \choose k} = \sum_{k=0}^n {n \choose k}1^{n-k}1^k = (1+1)^n = 2^n$$

Сочетания часто используются так же вот в каком ключе. Предположим, мама нам сказала: <<Пойди в магазин и купи $n$ каких-нибудь пирожков>>. Мы приходим в магазин, а там продаётся $k$ наименований пирожков. Сколько всего способов у нас есть удовлетворить мамин запрос? Задача в такой постановке приводит нас к понянию \term{сочетаний с повторениями}, поскольку из множества возможных элементов мы можем составлять наш набор, в котором какие-то элементы повторяются.

Для решения задачи предположим, что пирожки разного вида мы разложили по разным пакетам (я видел, что в ларьках у метро именно так часто и делают). Схематически мы будем разделять пакеты вертикальной чертой $|$, а пирожки (или, более общо, элементы множества), кружочками $\circ$. Для примера давайте считать, что у нас всего имеется $k=5$ видов пирожков, а купили мы $n=6$ пирожков, причем из них было 2 пирожка первого вида, три третьего и один четвертого. Остальных пирожков мы не покупали. На схеме это будет выглядеть как
$$|\circ||\circ\circ\circ|\circ||$$
Теперь мы можем догадаться, как подсчитать общее количество различных сочетаний с повторениями, если просто подсчитаем общее количество возможных схем такого вида. В схеме, приведенной выше, если отбросить крайние чёрточки $|$, получится $n+k-1$ различных позиций, на которых могут стоять чёрточки либо кружки. Причём мы точно знаем, что кружков всего будет $n$ штук, а чёрточек $k-1$. Чтобы получить какую-то конкретную схему, нам надо выбрать $n$ позиций под кружки, а остальные позиции мы занимаем чёрточками. Итого для личество сочетаний с повторениями мы имеем выражение
$${n+k-1 \choose n} = {n+k-1\choose k - 1}$$

\begin{exercise}
Докажите, что существует $2^{n-1}$ способов представить число $n$ в виде суммы ненулевых слагаемых. Задача довольно сложная, поэтому дам некоторые наводки. Вначале следует решить случай, когда имеется ровно $k$ слагаемых. Подход здест может быть аналогичным задаче с булочками выше, но надо учитывать то, что чтобы слагаемые были ненулевыми, мы не можем поставить подряд две черты, не поставив между ними кружок. Способов сделать это $n-1\choose k -1$ (докажите!). Отсюда уже довольно легко выводится и результат первоначальной задачи.
\end{exercise}

Давайте теперь опять решим задачу о количестве сочетаний, но теперь мы будем так же разбивать элементы нашего множества на группы. Например, мы можем опять рассматривать какие-то объекты, которые мы распределяем по ящикам. Пусть у нас изначально есть $n$ предметов. В первый ящик мы кладём $k_1$ предмет, во второй ящик $k_2$ предеметов и так далее до ящика под номером $m$, в которым ма кладём $k_m$ предметов. Причем мы в результате этого процесса должны разложить все предметы, то есть $\sum_{i=1}^m k_i = n$.

\begin{definition}
Число разбиений множества $[n]$ на множества мощностей $k_1, k_2, \ldots k_m$ называется мультиномиальным коэффициентом и обозначается как
$$n \choose k_1; k_2;\ldots; k_m$$
\end{definition}

\begin{thm}
$${n \choose k_1; k_2;\ldots; k_m} = \frac{n!}{k_1!k_2!\ldots k_m!}$$
\end{thm}
\begin{proof}
Выберём вначале $k_1$ элемент, спобовом сделать это $n\choose k_1$. Из оставшихся элементов теперь выберём во второе множество $k_2$ элементов, способов сделать это $n-k_1\choose k_2$. Продолжая рассуждать таким же образом, получаем
\begin{align*}
{n \choose k_1; k_2;\ldots; k_m} & = {n\choose k_1}{n-k_1\choose k_2}{n-k_1-k_2\choose k_3}\ldots{n-k_1-\ldots k_{m-1}\choose k_m} \\
&= \frac{n!}{k_1!(n-k_1)!}\cdot\frac{(n-k_1)!}{k_2!(n-k_1-k_2)!}\cdot\frac{(n-k_1-k_2)!}{k_3!(n-k_1-k_2-k_3)!}\cdot\ldots\\
&=\frac{n!}{k_1!k_2!\ldots k_m!}
\end{align*}
\end{proof}

\begin{thm}
$$(x_1+x_2\ldots x_m)^n = \sum_{k_1+\ldots + k_m = n}{n\choose k_1;k_2;\ldots;k_m}x_1^{k_1}x_2^{k_2}\ldots x_m^{k_m}$$
Здесь суммирование ведётся по всем возможным наборам чисел $\{k_i\}$, дающим в сумме $n$ (всего таких наборов по упражнению~3.62 будет $2^{n-1}$ штук).
\end{thm}
\begin{proof}
Аналогично доказательству теоремы 3.26. Проведите его самостоятельно.
\end{proof}

\begin{exercise}
Покажите, что теорема 3.26 является частным случаем для теоремы 3.28.
\end{exercise}

Завершим мы этот параграф довольно отвлечённым результатом, который показывается как связан возрастающий факториал с числами Стирлинга. Сделаем мы это главным образом для того, чтобы продемонстрировать новый для нас способ доказательства.

\begin{thm}
$$\sum_{k=0}^n \fstirling{n}{k}x^k = x^{\lceil n \rceil}$$
где $x$ ~--- некоторая переменная.
\end{thm}
\begin{proof}
В левой и правой частях (после раскрытия всех скобок) стоят многочлены, то есть выражения вида
$$a_n x^n + a_{n-1}x^{n-1} + \ldots + a_{0}$$
Величины $\{a_i\}$ называются коэффициентами многочлена. Мы покажем, что слева и справа в утверждении теоремы эти коэффициенты совпадают. Пусть
$$x^{\lceil n \rceil} = \sum_{k=0}^n a_{n, k} x^k$$
Это же самое можно переписать следующим образом:
\begin{align*}
x^{\lceil n \rceil} &= (x+n-1)x^{\lceil n-1 \rceil} = (x+n-1)\sum_{k=0}^{n-1}a_{n-1, k}x^k \\
& = \sum_{k=1}^n a_{n-1, k-1} x^k + (n-1)\sum_{k=0}^{n-1}a_{n-1,k} x^k \\
& = a_{n-1,0} + a_{n-1, n-1}x^n + \sum_{k=1}^{n-1} (a_{n-1,k-1} + (n-1) a_{n-1, k})x^k
\end{align*}
Из последнего выражения мы видим, что
$$a_{n, k} = a_{n-1, k-1} + (n-1)a_{n-1, k}$$
то есть что коэффициенты при $x$ в выражении $x^{\lceil n \rceil}$ удовлетворяют тому же рекурсивному тождеству, которое позволяет нам вычислять числа Стирлинга (теорема~3.20). Осталось доказать, что начальные значения так же совпадают, но это очевидно: при раскрытии скобок для $x^{\lceil n \rceil}$ получаем в явном виде, что $a_{n, n} = 1$ и $a_{n, 0} = 0$, что завершает доказательство.
\end{proof}